\section{Problem Statement and Background}
\label{S:problem}

\subsection{Notations}

\charlie{TODO: define Lagrangian}

$W_\pi(x, \lambda)$ is the Hessian of the Lagrangian at $(x, \lambda)$:

\begin{equation}
W_\pi(x, \lambda) = \pi \nabla^2 f(x) - \sum_{j=1}^m \lambda_j \nabla^2 c_j(x)
\end{equation}

We use superscript $k$ to denote iterates, such as $x^{(k)}$, and we use the convention
that $f^{(k)} := f(x^{(k)})$, $\nabla c^{(k)} := \nabla c(x^{(k)})$, $\nabla^2 f^{(k)} := \nabla^2 f(x^{(k)})$, etc.

\charlie{It makes formulae shorter, but I'm not sure we need these notations}

\subsection{Stationarity Conditions}

In general, we are only concerned with first-order stationary points. The optimality conditions of problem \eqref{eq:NLP}
are given by

\begin{align}\label{eq:KKT-feasibility}
\pi \nabla f(x^*) - \sum\limits_{j = 1}^m \lambda_j \nabla c_j(x^*) - \nu & = 0 \\ 
\underline{c_j} \le c_j(x^*) \le \overline{c_j} & \perp \lambda_j, \quad \forall j \in \{ 1, \ldots, m\} \label{eq:complementarity} \\
\underline{x_i} \le x^*_i \le \overline{x_i} & \perp \nu_i, \quad \forall i \in \{1, \ldots, n\} ,
\end{align}

where $\perp$ represents complementarity, namely \eqref{eq:complementarity} means

\begin{equation}
\begin{cases}
\lambda_j \ge 0, & \text{if } c_j(x^*) = \underline{c_j} \\
\lambda_j = 0, & \text{if } \underline{c_j} < c_j(x^*) < \overline{c_j} \\ 
\lambda_j \le 0, & \text{if } c_j(x^*) = \overline{c_j} .
\end{cases}
\end{equation}

If $\pi = 1$, then the optimality conditions are the well-known Karush-Kuhn-Tucker (KKT) conditions. If, on the other hand,
$\pi = 0$, these are the Fritz-John (FJ) conditions.

In general, we cannot assume that the nonlinear problem \eqref{eq:NLP} has a feasible point. Hence, nonlinear optimization
solvers must include provisions for infeasible problems. More generally, nonlinear solvers may converge to points that violate
standard constraint qualifications, and we must take these situations into account when defining optimality conditions.
If \eqref{eq:NLP} is infeasible, then we can define a feasibility problem for a partition $\mathcal{S}$,
$\overline{\mathcal{V}}$ and $\underline{\mathcal{V}}$ of the nonlinear constraints $\{1, \ldots, m\}$:

\todo{write this in a generic way, that can be specialized by filter or penalty}

\begin{equation}\label{eq:NLPinfeasibility}
\begin{array}{lll}
\underset{x}{\text{minimize}} 	& \sum\limits_{j \in \overline{\mathcal{V}}} c_j(x) - \sum\limits_{j \in \underline{\mathcal{V}}} c_j(x) & \\
\text{subject to} 				& \underline{c_j} \le c_j(x) \le \overline{c_j}, & \forall j \in \mathcal{S} \\
%								& \overline{c_j} \le c_j(x), & \forall j \in \overline{\mathcal{V}} \\
%								& c_j(x) \le \underline{c_j}, & \forall j \in \underline{\mathcal{V}} \\
								& \underline{x_i} \le x_i \le \overline{x_i}, & \forall i \in \{1, \ldots, n\}
\end{array}
\end{equation}

where we have partitioned the set of constraints into the set of feasible constraints $\mathcal{S}$, the set of
constraints violated at the upper bound $\overline{\mathcal{V}}$ and the set of constraints violated at the lower bound
$\underline{\mathcal{V}}$.

\charlie{I replaced the math definitions of the sets with a more general definition}
%\begin{align}
%\mathcal{S} 		& = \{j ~|~ \underline{c_j} \le c_j(x) + \nabla c_j(x)^T d \le \overline{c_j} \} \\
%\overline{\mathcal{V}} 	& = \{j ~|~ \overline{c_j} < c_j(x) + \nabla c_j(x)^T d \} \\
%\underline{\mathcal{V}} 	& = \{j ~|~ c_j(x) + \nabla c_j(x)^T d < \underline{c_j} \}
%\end{align}

The optimality conditions of the feasibility restoration problem \eqref{eq:NLPinfeasibility} are given by:

\begin{equation}\label{eq:KKT-infeasibility}
\begin{array}{ll}
\multicolumn{2}{l}{\sum\limits_{j \in \overline{\mathcal{V}}^*} \nabla c_j(x^*) - \sum\limits_{j \in \underline{\mathcal{V}}^*} \nabla c_j(x^*) - \sum\limits_{j \in \mathcal{S}(x^*)} \lambda_j \nabla c_j(x^*) - \nu = 0} \\ 
\underline{c_j} \le c_j(x^*) \le \overline{c_j} \perp \lambda_j, & \quad \forall j \in \mathcal{S}(x^*) \\
%\overline{c_j} \le c_j(x^*) & & \forall j \in \overline{\mathcal{V}}^* \\
%c_j(x^*) \le \underline{c_j} & & \forall j \in \underline{\mathcal{V}}^* \\
\underline{x_i} \le x^*_i \le \overline{x_i} \perp \nu_i, & \quad \forall i \in \{1, \ldots, n\}
\end{array}
\end{equation}

We note that other forms of feasibility restoration problem such as the minimum $\ell_2$ norm are possible. If the
multipliers of \eqref{eq:KKT-feasibility} are bounded by 1 in magnitude, then the resulting point is a stationary
point of the $\ell_1$-norm constraint minimization.

\subsection{Residuals}

In practice, we cannot hope to drive the error in the first-order conditions to zero, and hence, we adopt the
following approximate error conditions.

The KKT error is

\begin{equation}
\bigg\lVert \pi \nabla f(x^*) - \sum\limits_{j = 1}^m \lambda_j \nabla c_j(x^*) - \nu \bigg\rVert \le \varepsilon_d
\end{equation}

The complementarity error is

\begin{equation}
\bigg\lVert
\sum_{j: \lambda_j < 0} \lambda_j (c_j(x^*) - \overline{c_j}) +
\sum_{j: \lambda_j > 0} \lambda_j (c_j(x^*) - \underline{c_j}) +
\sum_{i: \nu_i < 0} \nu_i (x_i^* - \overline{x_i}) +
\sum_{i: \nu_i > 0} \nu_i (x_i^* - \underline{x_i})
\bigg\rVert \le \varepsilon_d
\end{equation}

The feasibility error for \eqref{eq:NLP} is

\begin{equation}
\sum\limits_{j=1}^m \bigg\lVert \max(0, \underline{c_j} - c_j(x^*), c_j(x^*) - \overline{c_j}) \bigg\rVert_1 \le \varepsilon_p
\end{equation}
\begin{equation}
\sum\limits_{j=1}^n \bigg\lVert \max(0, \underline{x_i} - x^*, x^* - \overline{x_i}) \bigg\rVert_1 \le \varepsilon_p ,
\end{equation}

for some primal and dual error constants $\varepsilon_p,  \varepsilon_d > 0$.
